
\chapter{Introduction}
%\begin{abstract}
The idea of generative models is one of the most promising approaches to Machine Learning tasks. Many such models have been used in recent years to solve the problem of inference of a posterior distribution, given some expert knowledge and assumptions on the prior distribution and using observed data to adapt these. The main obstacle in attempting to generate the true posterior is its computational intractability. Consequently, approximation methods have been employed, aiming to make computation more efficient while at the same time achieving a close estimate of the true posterior.

The paper focusses on the method of variational inference. Essentially, this converts the inference problem to an optimization one where, given a family of posterior approximations, we look for the distribution that is the closest to the true one. To measure this proximity between approximations and the true posterior, we use the Kullback-Leibler divergence. There has been lots of research in the classes of such, efficiently calculatable, approximation families, such as mean-field approximations, but all of these prove to be of limited expressiveness, leading to solutions that are far from the true posterior. In fact, there is strong evidence showing that more expressive classes of posteriors would result in better performance of variational inference, which is what motivates the work of the paper.

To address the above issue, the authors propose constructing the approximations through a normalizing flow, which applies a sequence of invertible transformations to a simple initial density until it achieves a desired level of complexity. The main types of flows considered and used in the experiments are:

\begin{enumerate}
	\item Invertible, linear-time flows, such as planar and radial flows.
	\item Finite, volume-preserving flows, such as the Non-linear Independent Components Estimation (NICE).
\end{enumerate}

The goal of both of the above is to provide a transformation to the original density that ends up being rich and faithful, while at the same time allowing for efficient computation of the determinant. The first kind of flow allows for computing the logdet-Jacobian in linear (in the number of dimensions) time, while the second one gives a Jacobian with a zero upper triangular part and therefore a determinant of 1. Thus, both can be efficiently implemented and used together with a Deep Latent Gaussian Model and an inference network to solve our problem.

% In addition to the practical implementation and the experiments, the paper gives proof that infinitesimal flows, which are normalizing flows whose length tends to infinity, can define a class of posteriors that is asymptotically able to recover the true posterior distribution, solving therefore one of the most important problems of variational inference. This is another good indicator that the method should perform well in the finite case.


- TODO: describe in more depth, after we have results

The paper’s experiments on the binarized MNIST dataset aim to demonstrate how variational inference with normalizing flows can outperform previous methods, compare the performance of different types of flows, and discuss the relation between the length of the flow and the quality of the resulting approximation.

- TODO: after we finish the extension

In our report, we replicate the original contributions of the paper by running the same experiments on MNIST. Additionally, we consider another class of flows, that of invertible convolution flows, which use the properties of circulant matrices to achieve efficient computation of the log-determinant. 

Finally, we provide an open-source implementation of the above in our Github repository: 
\url{https://github.com/ATML-Group-12/normalising_flows}

%\end{abstract}