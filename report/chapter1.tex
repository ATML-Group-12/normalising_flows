\section*{Normalizing Flows}
In the attempt to generate more complex families of posterior distributions, which should allow for better estimates of the true posterior, the paper introduces the idea of normalizing flows. A normalizing flow is a sequence of invertible transformations applied to a simple initial distribution (for example a unit gaussian), resulting in a complex distribution, whose density we can efficiently evaluate by inverting the flow and keeping track of the Jacobian of the composition of transformations, using the change of variable theorem.

\subsection*{Definition}

More formally, consider a member of the flow's sequence of transformations to be an invertible, smooth mapping $f:\mathbb{R}^d \rightarrow \mathbb{R}^d$. Let $\mathbf{y}=f(\mathbf{z})$ be the outcome of applying the transformation to a random variable $\mathbf{z}$ with distribution $q(\mathbf{z})$. Then, using the change of variable theorem, the resulting density is:
\[ q(\mathbf{y}) = q(\mathbf{z}) \left| \det \frac{\partial f^{-1}(\mathbf{y})}{\partial \mathbf{y}}  \right| = q(\mathbf{z}) \left| \det \frac{\partial f(\mathbf{z})}{\partial \mathbf{z}} \right| ^ {-1} \]
where the second equality follows from the inverse function theorem and the definition of $\mathbf{y}$.

We can then proceed to build more complicated densities by systematically composing multiple transformations such as the one above, since the composition of invertible transformations remains an invertible transformation. To simplify notation, we define the log-density $\ln q_K$ resulting from applying a sequence of $K$ transformations $f_1,f_2,...,f_K$ to a random variable $\mathbf{z_0}$ with an initial distribution $q_0$ as:
\[ \mathbf{z_K} = f_K \circ ... \circ f_2 \circ f_1(\mathbf{z_0}) \]
\[ \ln q_K(\mathbf{z_K}) = \ln q_0(\mathbf{z_0}) - \sum_{k=1}^{K} \ln \left| \det \frac{\partial f_k}{\partial \mathbf{z_{k-1}}} \right|  \]
where the first equation represents the sequence of random variables generated by the flow.  The normalizing flow is then defined as the path of the successive distributions $q_k$. 

\subsection*{Law of the Unconscious Statistician}
A useful property of these transformations is what is known as the law of the unconscious statistician (LOTUS). The LOTUS refers to the fact that we can evaluate expectations with respect to the transformed density $q_K$ without explicitly knowing it, by expressing any such expectation as:
\[ \mathbb{E}_{q_K} = \mathbb{E}_{q_0}[h(f_K \circ ... \circ f_2 \circ f_1(\mathbf{z_0}))] \]
This is an expectation over the known density $q_0$, which does not require computation of the logdet-Jacobian terms when $h(\mathbf{z})$ does not depend on $q_K$.

\subsection*{Invertible Linear-time Transformations}
Notice that a naive choice of these transformations would lead to a $O(d^3)$ complexity to compute the determinant of the Jacobian. Therefore, the classes of flows studied in our report are all based on the idea of having an efficient way to calculate this determinant. We begin by investigating two types of linear flows, namely planar and radial flows. 

\subsubsection*{Planar Flows}
Consider the following class of transformations:
\[ f(\mathbf{z}) = \mathbf{z}+\mathbf{u}h(\mathbf{w}^T\mathbf{z}+b) \]
where $\lambda = \{ \mathbf{w} \in \mathbb{R}^D, \mathbf{u} \in \mathbb{R}^D, b \in \mathbb{R} \}$ are free parameters and $h(\cdot)$ is a smooth, element-wise non-linearity. We can use the matrix determinant lemma to calculate the logdet-Jacobian term in linear time, yielding:
\[ \left| \det \frac{\partial f}{\partial \mathbf{z}} \right| = \left| 1+\mathbf{u}^T\psi(\mathbf{z})\right| \]
where $\psi(\mathbf{z}) = h'(\mathbf{w}^T\mathbf{z}+b)\mathbf{w}$ and $h'$ is the derivative of the function $h$. Using this, we can now substitute in the formula for $\ln q_K$ to get the following closed form for planar flows:
\[ \ln q_K (\mathbf{z_K}) = \ln q_0(\mathbf{z}) - \sum_{k=1}^K \ln \left| 1+\mathbf{u_k}^T\psi_k(\mathbf{z_{k-1}})\right| \]
The name planar comes from the fact that the above transformation applies a series of contractions and expansions in the direction perpendicular to the $\mathbf{w}^T\mathbf{z}+b = 0$ hyperplane to the initial density $q_0$.

\subsubsection*{Radial Flows}
The other family of invertible linear-time transformations studied is defined by the following equation:
\[f(\mathbf{z}) = \mathbf{z} + \beta h(\alpha,r)(\mathbf{z}-\mathbf{z_0}) \]
where $r=\left| \mathbf{z}-\mathbf{z_0}\right|$, $h(\alpha,r) = 1/(\alpha+r)$, and the set of parameters is $\lambda = \{ \mathbf{z_0}\in \mathbb{R}^D, \alpha \in \mathbb{R} ^{+}, \beta \in \mathbb{R} \}$. The time-complexity of the computation of the determinant of this class is also linear:
\[ \left| \det \frac{\partial f}{\partial \mathbf{z}} \right| = [1+\beta h(\alpha ,r)]^{d-1}[1+\beta h(\alpha ,r)+\beta h'(\alpha ,r)] \] % fix this equation
This transformation results in radial contractions and expansions around the reference point $\mathbf{z_0}$, hence the name radial flows.


